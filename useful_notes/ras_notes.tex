\documentclass[10pt, onecolumn]{article}
\usepackage[margin=1in]{geometry}
\usepackage{lmodern}% http://ctan.org/pkg/lm
\usepackage{authblk} % adds affiliations

\usepackage[utf8x]{inputenc}
\usepackage{nameref}
\usepackage[right]{lineno}
\usepackage{amsmath}
\usepackage{booktabs}
\usepackage[numbers,super]{natbib}
\usepackage{changepage}

% adjust caption style
\usepackage[aboveskip=1pt,labelfont=bf,
            labelsep=period,singlelinecheck=off]{caption}

% remove brackets from references
\makeatletter
\renewcommand{\@biblabel}[1]{\quad#1.}
\makeatother

\usepackage[colorinlistoftodos]{todonotes}

% headrule, footrule and page numbers
\usepackage{lastpage,fancyhdr,graphicx}
\usepackage{epstopdf}
\pagestyle{myheadings}
\pagestyle{fancy}
\fancyhf{}
\rfoot{\thepage/\pageref{LastPage}}
\renewcommand{\footrule}{\hrule height 2pt \vspace{2mm}}

% use \textcolor{color}{text} for colored text (e.g. highlight to-do areas)
\usepackage{color}

\definecolor{Gray}{gray}{.25}

\usepackage{graphicx}

% use if you want to put caption to the side of the figure
\usepackage{sidecap}

\usepackage{xcolor}
\usepackage[colorlinks = true,
            linkcolor = blue,
            urlcolor  = blue,
            citecolor = blue,
            anchorcolor = blue]{hyperref}

% ####################################################
% ####################################################
% ####################################################
\usepackage[colorinlistoftodos]{todonotes}
% ####################################################
% ####################################################
% ####################################################





% use for have text wrap around figures
\usepackage{wrapfig}
\usepackage[pscoord]{eso-pic}
\usepackage[fulladjust]{marginnote}
\reversemarginpar{}

\usepackage{gensymb}
\usepackage{siunitx}

% new commands
% q value
\newcommand{\qval}[1]{$q<10^{-#1}$}

% species names
\newcommand{\cel}{\emph{C.~elegans}}
\newcommand{\dicty}{\emph{D.~discoideum}}
\newcommand{\ecol}{\emph{E.~coli}}

% gene names
\newcommand{\gene}[1]{\mbox{\emph{#1}}}
\newcommand{\ras}{\gene{let-60} (\emph{ras})}
% \newcommand{\gene}[1]{\emph{#1}} # for MS word typesetting
\newcommand{\letgfn}{3,021}
\newcommand{\letlfn}{857}
\newcommand{\letgf}{\gene{let-60(gf)}}
\newcommand{\letlf}{\gene{let-60(lf)}}
\newcommand{\strongn}{2,821}
\newcommand{\weakn}{434}
\newcommand{\transn}{2,930}

% protein names

% DE genes numbers:

% downstream targets

% website commands


% more space between rows
\newcommand{\ra}[1]{\renewcommand{\arraystretch}{#1}}

\title{Useful notes on \ras{}}

\author[1,2]{David Angeles-Albores}
\author[1,2,*]{Paul W. Sterberg}
\affil[1]{Division of Biology and Biological Engineering, Caltech,
Pasadena, CA, 91125, USA}
\affil[2]{Howard Hughes Medical Institute, Caltech, Pasadena, CA, 91125, USA}
\affil[*]{Corresponding author. Contact: pws@caltech.edu}
\renewcommand\Affilfont{\itshape\small{}}

% document begins here
\begin{document}
% title
\maketitle
% author info
% \textbf{1} Division of Biology and Biological Engineering, Caltech,
% Pasadena, CA, 91125, USA\\
% \textbf{2} Howard Hughes Medical Institute, Caltech, Pasadena, CA, 91125, USA\\
% \textbf{*} Corresponding author. Contact: pws@caltech.edu


\linenumbers{}

\section*{Analysis of dominant-negative mutations of \ras{}}
See paper by Min Han and Sternberg~\cite{Han1991}. Briefly, Min Han found two
classes of dominant negative mutations: \emph{dn} mutations that would cause a
weakly penetrant Muv phenotype when injected into wild type animals and \emph{dn}
mutations that never cause a Muv phenotype when injected into wild type animals.

From studying one allele in each class, they show:
Non-Muv alleles:
\begin{enumerate}
  \item (sy99, s101), (sy92, sy95, sy100). Parentheses signify different strains
  containing the same mutations.
  \item Are lethal dominant. This lethality is strongly suppressed in a dosage-independent
  manner by \letlf{}.
  \item Vulval differentiation increases with decreasing \emph{dn} dosage:
  $dn/dn/+ < dn/+ < dn/+/+ = 1$.
\end{enumerate}

Muv alleles (AA 119 and 16):
\begin{enumerate}
  \item (sy93), (sy94)
  \item Are recessive viable, so $+/Df$ is dead.
  \item Are toxic to wild-type product, so vulval differentiation looks strange:
  $dn/+ < dn/dn \sim dn/+/+ < wt < dn/dn/+$.
  \item In $dn/dn/+$ animals, signaling is partially signal-independent.
\end{enumerate}

\section*{sur-1/mpk-1 acts downstream of \ras{}}
See Yan Wu and Min Han~\cite{Wu1994}. 10\% of \gene{sur-1} (ku1) homozygotes
have a P6.p non-differentiation phenotype. \gene{sur-1} is partially dominant
over \ras{}. Null homozygotes are dead.


\section*{\gene{ksr-1} encodes a novel raf-related kinase involved in Ras-mediated
signal transduction}
See paper by Sundaram and Han~\cite{Sundaram1995}. \gene{ksr-1} appears to be a
parallele signal transduction pathway to \gene{lin-45} (Raf) and MAPK. Interestingly
\gene{ksr-1} mutants do not have obvious vulval phenotypes by themselves, but
loss-of-function mutants suppress \letgf{} phenotypes. Double mutants of
\gene{ksr-1} and MAPK are synthetically dead.

\section*{A Ras-mediated signal transduction pathway is involved in control of sex
myoblast migration in \cel{}}
See paper by Sundaram and Han~\cite{Sundaram1996}. \letlf{} have posteriorly
positioned sex myoblasts (SM) relative to wild-type. \letgf{} mutants also have
posteriorly positioned SMs but the phenotype is less severe. Similarly, \gene{ksr-1}
mutants have posterior SMs. Loss of \gene{egl-15}, an FGF receptor, turns gonad from
attractive to repulsive. The \gene{ksr-1; egl-15} double has the same phenotype
as the \gene{egl-15} single.

$
egl-15; ksr-1 = egl-15 < ksr-1 < WT
$

\letgf{} causes a weak (10\%) SM migration phenotype (posterior). Unpublished:
\letgf{} coupled with \gene{lin-45}, \gene{ksr-1}, \gene{mek-2} or \gene{sur-1}
dramatically enhances SM positionining defects. Surprisingly, increasing
\ras{} function by injection of wild-type or \letgf{} genes into a +/+ background
does not cause positioning defects.

Increased \ras{} activity partially suppresses \gene{egl-15} or \gene{egl-17}
phenotypes. n1046gf allele can't suppress, but transgenes can. n1046gf also does
not suppress \gene{sem-5} migratory defects, whereas injected wild-type transgenes
do suppress (but the GF allele does NOT suppress [they argue it does, but small
numbers]).


\section*{SUR-5 Negatively Regulates \ras{} activity during Vulval induction}
See Gu and Han~\cite{Gu1998}.
\gene{sur-5} does not suppress all \emph{dn} mutants. It does not suppress
\emph{sy93} and \emph{sy100} (no association with Muv/nonMuv classes).

Take-home: \gene{sur-5} genetically inhibits \ras{} and the mutations suggest
there may be multiple modes of interaction between \ras{} and effectors.

\section*{SUR-8 Positively Regulates RAS-Mediated signaling in \cel{}}
See Sieburth and Han~\cite{Sieburth1998}. \gene{sur-8}, a conserved Ras-binding protein acts
to promote \ras{} (found via suppressor screen of \letgf{}), although
\gene{sur-8} by itself has no phenotype. \gene{sur-8} is synthetically lethal
with other downstream \ras{} effectors. \gene{sur-8(ku167); ksr-1(ku68)} has a
P6.p non-induction phenotype at a 10\% rate, identical to \gene{sur-1}.

\section*{The Ras-MAPK pathway is important for olfaction in \cel{}}
See paper by Hirosu and Iino~\cite{Iino2000}. Note that there is an important corrigendum
to this paper. In theory, \ras{} is shown to be important for AWA- and
AWC-mediated attraction to odors. Notably, gf and lf mutations are said to affect
odor detection in the same direction. However, the results are weak and not
satisfactory to believe in (to my mind).

\section*{SOS-1 is necessary for multiple RAS signals}
See paper by Chieh-Chang and Sternberg~\cite{Chang2000}. A very confusing paper, albeit
with a lot of data. Need to read again. SOS-1 is ID'ed as the GEF for \ras{} in
many situations.

\section*{EGL-15 Signaling Pathway Implicates\ldots{} in FGF Signal Transduction}
See paper by Schutzman and Stern~\cite{Schutzman2001}. \letgf{} suppresses a a Clr phenotype
(Soc), similarly to \gene{egl-15(gf)}. They found a gain of function mutation
(G60R). This mutant is partially Muv and has a protruding excretory pore. As
hets, these animals are Clr at 25\degree{}C, though normal at 15\degree{}C.

Strangely, the canonical gf mutation, \emph{n1046}, does not show a Clr phenotype.

Claim is that EGL-15 acts through SEM-5 in one pathway or through SOC-1/PTP-2
in another to prevent the Clr phenotype. PTP-2 and SOC-1 may talk to RAS themselves.

\section*{EOR-1 and EOR-2 regulate Ras and Wnt redundantly with SUR-2(Mediator)}
See paper by Howard and Sundaram~\cite{Howard2002}. In this paper, the claim is that
\gene{eor-1} and \gene{eor-2} act downstream of Ras and Wnt to modulate signal
from these things. They show a number of synthetic effects, namely rod-like
lethality with Ras pathway components, though evidence for involvement in Ras
signaling is weak when considering Ras pathway loss-of-function (almost no Vul).
However, Ras gain-of-function mutations leading to Muv phenotypes are strongly
attenuated by mutations in \gene{eor-1} and \gene{eor-2}. This is true even
for \gene{lin-1}, suggesting these genes act downstream of Ras.

Somewhat surprisingly, \gene{eor-1} and \gene{eor-2} are unable to suppress even
partially the synMuv phenotype of \gene{lin-15}, although they do inhibit the
0 P11.p phenotype of \gene{lin-15}.

Evidence for interaction with Wnt is mainly based on suppression of a 2 P11.p
phenotype of \gene{pry-1}, but no other phenotype.

\gene{lin-1} is claimed to act both positively and negatively with the Ras pathway
because although \gene{lin-1(lf)} mutants are alive, double mutants with either
\gene{eor-1} or \gene{eor-2} are completely dead; whereas maternally rescued
doubles have a double Muv/Vul phenotype reminiscent of \gene{lin-25}. Based on
these two phenotypes, the claim is that \gene{lin-1} acts bidirectionally on
the Ras pathway. Another paper, by Tiensuu and Tuck~\cite{}, further addresses
this issue.

\section*{\gene{sur-2} functions late in \gene{let-60} signaling}
See paper by Singh and Han~\cite{}. \gene{sur-2} attenuates the Muv phenotype of
the \letgf{} mutant. Mutations in this gene lead to Vul and Mab phenotypes.

\section*{\gene{lin-1} has positive and negative function in Ras signaling}
See paper by Tiensuu~\cite{Tiensuu2005}
In the absence of \gene{eor-1}, \gene{lin-1} mutants are dead unless maternally
rescued for LIN-1 protein, in which case they show a Muv/Vul dual phenotype.
Larval rod-like death is typical of \ras{} loss-of-function, whereas the Muv
phenotype is reminiscent of \letgf{} mutants.

This paper goes on to claim, using \gene{egl-17::gfp} expression as a proxy,
that \gene{lin-1} has positive and negative effects on the Ras pathway. However,
all of their epistatic evidence is entirely consistent with an `additive' model,
in which \gene{lin-1} represses \gene{egl-17} and none of the other genes affect
\gene{egl-17} expression. Particularly worrisome is the lack of dynamic range
(almost everything is 0 or 100\%).

The most interesting aspect of this paper us that in \gene{lin-1} mutants, the
excretory duct cell is duplicated (as defined by expression of a
\gene{lin-48::gfp} reporter), whereas in a \gene{eor-1} mutant the duct cell
appears wild-type. In the double mutant, however, the duct cell appears to be
missing, which also happens to be the phenotype of a \letgf{} mutant.

In general, a very poorly written paper with an overabundance of weak data.

\section*{A gain-of-function allele of \gene{cbp-1} (p300) increase Histone
          Acetyltransferase activity and antagonizes Ras}
See Eastburn and Han~\cite{Eastburn2005}. More Mediator-associated associations with Ras.

\section*{ISWI and NURF301 antagonize Rb-like pathways in multiple cell fates}
See Andersen and Horvitz~\cite{}. \gene{isw-1} suppresses a number of class B
phenotypes, as well as the synMuv phenotype for a number of AB doubles.

\section*{Non-cell-autonomous role of Ras in neuroblast determination}
See paper by Parry and Sundaram~\cite{Parry2014}. Paper shows that the G1 cell, which
helps generate the excretory pore initially, but later becomes a neuroblast,
requires \ras{} for its maturation non-cell autonomously.

\section*{KSR forms a Multiprotein Signaling Complex and Modulates MEK
          localization}
See paper by Stewart, Sundaram and Guan~\cite{Stewart1999}. Nice and thorough paper showing
that KSR binds MEK in an enormous 700kDA complex (KSR and MEK1/2 <50kDA each).
\cel{} \gene{ksr} alleles cannot bind MEK1. KSR did not appear to affect MEK
activity or activation, but it does appear to localize MEK1 to the membrane.

\section*{KSR is a scaffold required for MAPK activation}
See paper by Roy and Therrien~\cite{}. Fruitfly paper. Claim is that KSR is a
scaffold that facilitates phosphorylation of MEK by RAF.

\section*{PP2A positively regulates Ras-signaling}
See paper by Sierburth and Han~\cite{}. They find that PP2A acts positively to
promote Ras signaling in a common pathway with \gene{ksr-1}, probably upstream
of \gene{raf}.

\section*{Wnt signaling bypasses Ras in vulval induction}
See paper by Gleason and Eisenman~\cite{}. Take-home message is that
\gene{lin-39}, a Hox gene, is controlled positively by both Wnt and Ras. When
Wnt is overloaded, \gene{lin-39} activity becomes independent of Ras, as assayed
by a \gene{pry-1; let-60(n1531dn)} double mutant. At 15\degree{}C, the effect is
particularly pronounced.

\section*{\gene{ksr-1} and \gene{ksr-2} have unique and redundant functions and
          are required for MPK-1 ERK phosphorylation}
See beautiful paper by Ohmachi and Sundaram~\cite{Ohmachi2002}. This paper explains that
\gene{ksr-1} and \gene{ksr-2} are part of a scaffold that promotes the
phosphorylation of MPK-1. Whereas \gene{ksr-1} is required for SM and dispensable
for almost everything else, \gene{ksr-2} is required for Ras signaling in the
germline meiotic progression. Morevoer, knocking both of these genes out results
in a significant loss of activated MPK-1 and phenotype strongly reminiscent of
\letlf{} mutants, as assayed by vulval induction, SM migration defects and
fertility.

Interestingly, KSR1/2 were both identified as synthetic interactors with
\letgf{} and said to be in a `redundant' pathway. However, some of these papers
suggest that they were not so much redundant pathways as concentration hubs
for multiple Ras-related pathways. This is a fantastic paper.

\section*{PTEN Negatively Regulates MAPK Signaling during
          \cel{} Vulval Development}
Nakdimon and Durbin~\cite{Nakdimon2012}. \gene{daf-2} promotes Ras, \gene{daf-18} inhibits.
\gene{daf-2} allegedly talks to \gene{sem-5} and \gene{age-1}. AGE-1 promotes
PIP$_3$ formation, which in turn promotes Ras signaling at an unknown location.
Meanwhile, DAF-18 promotes PIP$_3$ degradation, thus inhibiting the Ras pathway.
However, DAF-18 also appears to act in a PIP$_3$-independent pathway to inhibit
MAPK (MPK-1) activity.

\section*{PUF-8 negatively regulates RAS/MAPK signalling to promote
         differentiation of \cel{} germ cells}
Vaid and Subramaniam~\cite{Vaid2013}. GLP-1 (Notch) promotes mitotic proliferation by
blocking exit into meiosis. PUF-8 and GAP-3 (a Ras GAP protein) appear to function
redundantly to block LET-60 in the mitotic and meiotic entry regions.

Phosphorylated MPK-1 is
present in the transition to proximal parts of the gonad and repressed distally.
However, in \gene{puf-8;gap-3} double mutants, MPK-1 is phosphorylated throughout
and leads to a constitutively mitotic germline.
The authors show that RNAi knockdown of \gene{let-60} restores meiosis to the
germline. This would appear to show that \gene{let-60} inhibits germline
maturation.

In general, this paper is best understood with a view of \gene{let-60} as a
proliferative factor. It promotes mitosis by blocking entry into meiosis in the
distal germline, for which reason it is repressed there. Overabundance of LET-60
thus leads to tumor formation.





\bibliography{citations}

%This defines the bibliographies style.
\bibliographystyle{naturemag}

\end{document}
